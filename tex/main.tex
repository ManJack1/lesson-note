% !TEX program = xelatex
% !TEX options = --shell-escape
\documentclass[10pt, a4paper, oneside, UTF8]{ctexbook}
\usepackage{amsmath, amsthm, amssymb, bm, graphicx, hyperref, mathrsfs}
\usepackage[dvipsnames]{xcolor}
\usepackage{tikz}
\usetikzlibrary{backgrounds,arrows,shapes,tikzmark,calc}
\usepackage{geometry}
\usepackage{annotate-equations}
\usepackage{extarrows}
\usepackage{thmbox}
\usepackage{svg}
\usepackage{fancyhdr}
\usepackage{titlesec}
\usepackage{setspace}
\usepackage{enumitem}
\usepackage{caption}

% Custom commands and environments
\newcommand{\diff}{\mathrm{d}}

\newenvironment{note}
{\par\textcolor{blue}{\bfseries Note:}\itshape}{\par}
\newenvironment{remark}
{\par\textcolor{blue}{\bfseries Remark:}\itshape}{\par}

\newtheorem{theorem}{Theorem}[section]
\newtheorem{lemma}[theorem]{Lemma}
\newtheorem{definition}{Definition}[section]
% \newtheorem{example}{Example}[section]
\newtheorem{proposition}{Proposition}[section] % 新添加的命题环境
\newtheorem{corollary}{Corollary}[section] % 新添加的推论环境
\renewcommand{\eqnannotationfont}{\bfseries\small}

% Geometry settings (consolidated)
\geometry{
  a4paper,
  total={170mm,257mm},
  left=20mm,
  right=20mm,
  top=20mm,
  bottom=20mm,
}

% Page style
\fancyhf{}
\fancyhead[L]{\small\leftmark}
\fancyhead[R]{\small\thepage}
\pagestyle{fancy}

% Title formatting
\titleformat{\chapter}[hang]
{\normalfont\Large\bfseries}
{\thechapter.}{1em}{}
\titlespacing*{\chapter}{0pt}{*1.5}{*1}

% Section and subsection formatting
\titleformat{\section}
{\normalfont\large\bfseries}
{\thesection}{1em}{}
\titlespacing*{\section}{0pt}{*1.5}{*1}

\titleformat{\subsection}
{\normalfont\normalsize\bfseries}
{\thesubsection}{1em}{}
\titlespacing*{\subsection}{0pt}{*1.5}{*1}

% Line spacing
\setstretch{1.1}

% Paragraph spacing
\setlength{\parskip}{0.5em} % 段落间距
\setlength{\parindent}{1.5em} % 段落缩进

% Itemize settings
\setlist{noitemsep, topsep=0pt, left=2em}

% Caption settings
\captionsetup{font=small, labelfont=bf}

\title{{\Huge{\textbf{数学分析}}}\\------张筑生}
\author{ManJack}
\date{\today}
\linespread{1.5}

\begin{document}

\maketitle

\pagenumbering{roman}
\setcounter{page}{1}

\newpage
\begin{center}
  \Huge\textbf{前言}
\end{center}

这是数学系线性代数的笔记,写给自己。如有错误请见谅,这些只是作为分享。

\begin{flushright}
  \begin{tabular}{c}
    ManJack \\
    \today
  \end{tabular}
\end{flushright}

\newpage
\tableofcontents
\newpage
\pagenumbering{arabic}
\setcounter{page}{1}

\chapter{Linear Equations}

% Start your content here

\section*{线性方程组的解法}

\subsection*{解线性方程组的矩阵消元法}

\begin{example}
  将前面系数提取出来记录为:
  \[
    \begin{cases}
      x_1+3x_{2}+x_3=2 \\
      3x_1+4x_2+2x_3 = 9 \\
      -x_1-5x_{2}+4x_{3} = 10\\
      2x_{1}+7x_{2} +x_{3}= 1
    \end{cases}
    \longrightarrow
    \begin{bmatrix}
      1 &3&1&2\\
      3&4&2&9\\
      -1&-5&4&10\\
      2&7&1&1
    \end{bmatrix}
  \]
  将上述例子化简后的结果为:
  \[
    \begin{bmatrix}
      1 & 3 & 1 & 2 \\
      0 & 1 & -1 & -1 \\
      0 & 0 & 3 & 6 \\
      0 & 0 & 0 & 0 \\
    \end{bmatrix}
  \]
\end{example}

矩阵的初等行变换:
考虑一个简单的二元线性方程组:
\[
  \begin{cases}
    a_{11}x_1+a_{12}x_2 = b_1 \\
    a_{21}x_1+a_{22}x_2 = b_2 \\
  \end{cases}
\]
我们可以将其增广矩阵化为:
\[
  \begin{bmatrix}
    a_{11} & a_{12} & b_1 \\
    a_{21} & a_{22} & b_2 \\
  \end{bmatrix}
  \xrightarrow{R_2 \leftarrow R_2 - \frac{a_{21}}{a_{11}} R_1}
  \begin{bmatrix}
    a_{11} & a_{12} & b_1 \\
    0 & a_{22} - \frac{a_{21}}{a_{11}} a_{12} & b_2 -
    \frac{a_{21}}{a_{11}} b_1 \\
  \end{bmatrix}
\]
这里假设 $a_{11} \neq 0$。

\subsection*{行列式}

\subsubsection*{逆序数}

\begin{theorem}
  对换改变n元排列逆序数的奇偶性。
\end{theorem}

\begin{proof}
  首先考虑相邻元素对换的情况。若序列 $n$ 的排列为:
  \[
    n = (\dots, i, j, \dots)
  \]
  对换 $i, j$ 后得到:
  \[
    n_{1} = (\dots, j, i, \dots)
  \]
  则逆序数 $\tau$ 的变化只有两种情况:
  \[
    \begin{cases}
      \tau(n_{1}) = \tau(n) - 1 & \text{如果 } (i, j) \text{ 是一个逆序对} \\
      \tau(n_{1}) = \tau(n) + 1 & \text{如果 } (i, j) \text{ 不是一个逆序对}
    \end{cases}
  \]
  在这两种情况下,逆序数的奇偶性都发生了改变。

  接下来考虑相距较远元素对换的情况。若 $n$ 的排列为:
  \[
    n = (\dots, i, \overbrace{\dots}^{k \text{个元素}}, j, \dots)
  \]
  我们可以通过一系列相邻对换来实现 $i$ 和 $j$ 的对换。首先,将 $j$ 向左移动 $k+1$ 次,越过 $k$ 个中间元素和元素 $i$,得到:
  \[
    (\dots, j, i, \dots, \dots)
  \]
  这一步进行了 $k+1$ 次相邻对换。然后,将 $i$ 向右移动 $k$ 次,越过它右边的 $k$ 个中间元素,到达原来 $j$ 的位置,得到:
  \[
    (\dots, j, \overbrace{\dots}^{k \text{个元素}}, i, \dots)
  \]
  这一步进行了 $k$ 次相邻对换。
  总共进行了 $k+1+k = 2k+1$ 次相邻对换。由于 $2k+1$
  是奇数,每次相邻对换都改变奇偶性,所以总共进行了奇数次奇偶性改变,最终排列 $n_1$ 的逆序数 $\tau(n_1)$
  的奇偶性与原排列 $n$ 的逆序数 $\tau(n)$ 相反。
\end{proof}

\begin{theorem}
  若对一个标准顺序排列($1, 2, \dots, n$)的n元排列进行k次对换得到新的排列,则新排列逆序数的奇偶性与k的奇偶性相同。
\end{theorem}
\begin{proof}
  标准排列的逆序数为 0(偶数)。根据上一个定理,每次对换改变逆序数的奇偶性。进行 $k$ 次对换后,逆序数的奇偶性改变了 $k$
  次。因此,最终排列的逆序数的奇偶性与 $k$ 的奇偶性相同。
\end{proof}

\subsubsection*{n阶行列式的定义}

\begin{definition}
  设 $A = (a_{ij})$ 是一个 $n \times n$ 矩阵。其行列式定义为:
  \[
    \det(A) = |A| = \left|
    \begin{matrix}
      a_{11} & a_{12} & \cdots & a_{1n} \\
      a_{21} & a_{22} & \cdots & a_{2n} \\
      \vdots & \vdots & \ddots & \vdots \\
      a_{n1} & a_{n2} & \cdots & a_{nn}
    \end{matrix}\right| = \sum_{j_1j_2 \cdots j_n} (-1)^{\tau(j_1j_2
    \cdots j_n)} a_{1j_1}a_{2j_2}\cdots a_{nj_n}
  \]
  其中 $j_1j_2 \cdots j_n$ 是 $1, 2, \dots, n$ 的一个排列,$\tau(j_1j_2 \cdots
  j_n)$ 是该排列的逆序数,求和遍及所有 $n!$ 个不同的排列。
\end{definition}

n阶的行列式记作 $|A|$ 或 $\det(A)$。

主对角线下方全为0的行列式称为\textbf{上三角行列式}。主对角线上的元素称为主对角线元素。
\[
  A =
  \begin{vmatrix}
    a_{11} & a_{12} & a_{13} & \cdots & a_{1n} \\
    0      & a_{22} & a_{23} & \cdots & a_{2n} \\
    0      & 0      & a_{33} & \cdots & a_{3n} \\
    \vdots & \vdots & \vdots & \ddots & \vdots \\
    0      & 0      & 0      & \cdots & a_{nn}
  \end{vmatrix}
\]
根据行列式的定义 $\det(A) = \sum_{j_1j_2 \cdots j_n} (-1)^{\tau(j_1j_2 \cdots
j_n)} a_{1j_1}a_{2j_2}\cdots a_{nj_n}$,考虑求和中的每一项
$a_{1j_1}a_{2j_2}\cdots a_{nj_n}$。
如果这一项非零,则必须满足 $a_{ij_i} \neq 0$ 对所有 $i=1, \dots, n$ 成立。
对于上三角矩阵:
\begin{itemize}
  \item 从最后一行 $i=n$ 开始,由于 $a_{n k} = 0$ 对 $k < n$,只有当 $j_n=n$ 时
    $a_{nj_n}$ 才可能非零。
  \item 考虑倒数第二行 $i=n-1$。由于 $a_{(n-1)k} = 0$ 对 $k < n-1$,且 $j_n=n$
    已被选定,所以 $j_{n-1}$ 只能是 $n-1$ (因为 $j_{n-1} \neq j_n=n$),才可能使
    $a_{(n-1)j_{n-1}}$ 非零。
  \item 以此类推,从下往上,第 $i$ 行的列指标 $j_i$ 必须满足 $j_i \ge i$。同时,由于 $j_1,
    \dots, j_n$ 是 $1, \dots, n$ 的一个排列,它们必须互不相同。结合这两个条件,唯一可能使乘积项非零的排列是
    $j_1=1, j_2=2, \dots, j_n=n$。
\end{itemize}
\[
  \begin{cases}
    a_{nj_n} \neq 0 \implies j_n = n \\
    a_{(n-1)j_{n-1}} \neq 0 \implies j_{n-1} = n-1 \text{ (因为 }
    j_{n-1} \neq j_n \text{)} \\
    \vdots \\
    a_{1j_1} \neq 0 \implies j_1 = 1
  \end{cases}
\]
这个排列是 $1, 2, \dots, n$,其逆序数 $\tau(12\dots n) =
0$。因此,行列式定义中的求和只有一项是非零的,即对应于主对角线元素的乘积。
所以,上三角行列式的值为:$\det(A) = (-1)^0 a_{11}a_{22}\cdots a_{nn} =
\prod_{i=1}^{n} a_{ii}$。

\paragraph{命题1}
给定行指标的一个排列 $i_1, i_2, \dots, i_n$,$A$ 的行列式也可以表示为:
\[
  \det(A) = \sum_{k_1k_2 \cdots k_n} (-1)^{\tau(i_1i_2 \cdots i_n) +
  \tau(k_1k_2 \cdots k_n)} a_{i_1k_1}a_{i_2k_2}\cdots a_{i_nk_n}
\]
其中求和遍及列指标的所有排列 $k_1k_2 \cdots k_n$。

\begin{proof}
  考虑行列式定义中的任意一项 $(-1)^{\tau(j_1j_2 \cdots j_n)}
  a_{1j_1}a_{2j_2}\cdots a_{nj_n}$。
  我们可以将因子 $a_{1j_1}a_{2j_2}\cdots a_{nj_n}$ 按照给定的行指标排列 $i_1, i_2,
  \dots, i_n$ 重新排序。设 $a_{i_1k_1}a_{i_2k_2}\cdots a_{i_nk_n}$ 是
  $a_{1j_1}a_{2j_2}\cdots a_{nj_n}$ 重新排序后的形式。
  这意味着,当 $i_p=r$ 时,必然有 $k_p=j_r$。也就是说,$(i_1, k_1), (i_2, k_2), \dots,
  (i_n, k_n)$ 与 $(1, j_1), (2, j_2), \dots, (n, j_n)$ 是相同的一组成对下标,只是顺序不同。
  将行指标排列 $i_1i_2 \cdots i_n$ 通过 $s = \tau(i_1i_2 \cdots i_n)$
  次相邻对换恢复成标准顺序 $1, 2, \dots, n$,同时对列指标 $k_1k_2 \cdots k_n$
  进行同样的伴随对换,会得到排列 $j_1j_2 \cdots j_n$。
  根据排列理论,两个排列同时进行相同的 $s$ 次对换后,它们的逆序数奇偶性之和保持不变,或者说 $\tau(j_1j_2 \cdots
  j_n)$ 的奇偶性与 $\tau(i_1i_2 \cdots i_n) + \tau(k_1k_2 \cdots k_n)$ 的奇偶性相同。
  即 $(-1)^{\tau(j_1j_2 \cdots j_n)} = (-1)^{\tau(i_1i_2 \cdots i_n) +
  \tau(k_1k_2 \cdots k_n)}$。
  因此,
  \[
    (-1)^{\tau(j_1j_2 \cdots j_n)} a_{1j_1}a_{2j_2}\cdots a_{nj_n} =
    (-1)^{\tau(i_1i_2 \cdots i_n) + \tau(k_1k_2 \cdots k_n)}
    a_{i_1k_1}a_{i_2k_2}\cdots a_{i_nk_n}
  \]
  对所有排列求和,即可得到两种定义形式是等价的:
  \[
    \sum_{j_1j_2 \cdots j_n} (-1)^{\tau(j_1j_2 \cdots
    j_n)}a_{1j_1}a_{2j_2}\cdots a_{nj_n} = \sum_{k_1k_2 \cdots k_n}
    (-1)^{\tau(i_1i_2 \cdots i_n)+\tau(k_1k_2 \cdots
    k_n)}a_{i_1k_1}a_{i_2k_2}\cdots a_{i_nk_n}
  \]
  这证明了命题。这个命题说明行列式也可以按任意行(或类似的可以证明按任意列)展开,符号取决于行排列和列排列的逆序数。
\end{proof}

\[
  \det(A)
.\]
\end{document}
